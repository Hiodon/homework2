\documentclass{tufte-handout}

\title{Python 101}

\author[The Academy]{Ilya Michurin}

%\date{28 March 2010} % without \date command, current date is supplied

%\geometry{showframe} % display margins for debugging page layout

\usepackage{graphicx} % allow embedded images
  \setkeys{Gin}{width=\linewidth,totalheight=\textheight,keepaspectratio}
  \graphicspath{{graphics/}} % set of paths to search for images
\usepackage{amsmath}  % extended mathematics
\usepackage{booktabs} % book-quality tables
\usepackage{units}    % non-stacked fractions and better unit spacing
\usepackage{multicol} % multiple column layout facilities
\usepackage{lipsum}   % filler text
\usepackage{fancyvrb} % extended verbatim environments
  \fvset{fontsize=\normalsize}% default font size for fancy-verbatim environments
  
  
  
    %MADNESS
  
  \usepackage[T1]{fontenc} % Use 8-bit encoding that has 256 glyphs
\usepackage{fourier} % Use the Adobe Utopia font for the document - comment this line to return to the LaTeX default
\usepackage[english]{babel} % English language/hyphenation
\usepackage{amsmath,amsfonts,amsthm} % Math packages
\usepackage{mathtools}% http://ctan.org/pkg/mathtools
\usepackage{etoolbox}% http://ctan.org/pkg/etoolbox
\usepackage{lipsum} % Used for inserting dummy 'Lorem ipsum' text into the template
\usepackage{units}% To use \nicefrac
\usepackage{cancel}% To use \cancel
%\usepackage{physymb}%To use r
\usepackage{sectsty} % Allows customizing section commands
\usepackage[dvipsnames]{xcolor}
\usepackage{pgf,tikz}%To draw 
\usepackage{pgfplots}%To draw 
\usetikzlibrary{shapes,arrows}%To draw 
\usetikzlibrary{patterns,fadings}
 \usetikzlibrary{decorations.pathreplacing}%To draw curly braces 
 \usetikzlibrary{snakes}%To draw 
 \usetikzlibrary{spy}%To do zoom-in
 \usepackage{setspace}%Set margins and such
 %\usepackage{3dplot}%To draw in 3D
\usepackage{framed}%To get shade behind text



\definecolor{shadecolor}{rgb}{0.9,0.9,0.9}%setting shade color
\allsectionsfont{\centering \normalfont\scshape} % Make all sections centered, the default font and small caps
  
  

  
  

% Standardize command font styles and environments
\newcommand{\doccmd}[1]{\texttt{\textbackslash#1}}% command name -- adds backslash automatically
\newcommand{\docopt}[1]{\ensuremath{\langle}\textrm{\textit{#1}}\ensuremath{\rangle}}% optional command argument
\newcommand{\docarg}[1]{\textrm{\textit{#1}}}% (required) command argument
\newcommand{\docenv}[1]{\textsf{#1}}% environment name
\newcommand{\docpkg}[1]{\texttt{#1}}% package name
\newcommand{\doccls}[1]{\texttt{#1}}% document class name
\newcommand{\docclsopt}[1]{\texttt{#1}}% document class option name
\newenvironment{docspec}{\begin{quote}\noindent}{\end{quote}}% command specification environment
\begin{document}

\maketitle % Print the title section


\normalsize

\vspace{1cm}

\vspace{1cm}


\section{Main Menu - Program Code}

\marginnote[40pt]{
First code changes your number from base 4 to base 2

we have to give a definition to the factors that we use in this, c for number, b for base, and etc.

First, we write that our number has to be bigger than a converted base 

After, step by step, we are giving a definition to a new factors that at the end will show us a new number in the base that we chose to convert

But in the line ans=str(n)+ans we want our answer to be a string

At the end we need a basic command print ans to show us our answer, therefore our converted number from Base 4 to Base 2 

Second code is used to convert numbers from Base 4 to Base 10

Actually, this code is pretty same than one that we had before, but here we give a definition L with command len() that means we return the length (the number of items)

To find the answer we have to take our number as an integer 

And after we right a formula for our exponent 

The easiest and the last step is to print ans 

And work is done


}

\begin{framed}
\begin{verbatim}
c=5112
a=5112
b=2
e=0
ans=""
while c>(b**e):
    m=a%(b**(e+1))
    n=m/(b**e)
    ans=str(n)+ans
    a=a-m
    e=e+1
print ans

c="3111"
l=len(c)
b=4
e=0
ans=0
while e<l:
    ans=ans+int(c[l-1-e])*(b**e)
    e=e+1
print ans


\end{verbatim}
\end{framed}

\marginnote[40pt]{This is the output of the lines of code above.}

\begin{shaded}
\begin{verbatim}

Enter base:  4
Enter Number to convert:  135
135 becomes  2 0 1 3 

\end{verbatim}
\end{shaded}



\vspace{1cm}

\bibliography{sample-handout}
\bibliographystyle{plainnat}



\end{document}